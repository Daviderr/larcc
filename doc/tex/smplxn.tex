% -------------------------------------------------------------------------
% ------ nuweb macros (redefine as desired, or omit with "nuweb -p") ------
% -------------------------------------------------------------------------
\providecommand{\NWtxtMacroDefBy}{Macro defined by}
\providecommand{\NWtxtMacroRefIn}{Macro referenced in}
\providecommand{\NWtxtMacroNoRef}{Macro never referenced}
\providecommand{\NWtxtDefBy}{Defined by}
\providecommand{\NWtxtRefIn}{Referenced in}
\providecommand{\NWtxtNoRef}{Not referenced}
\providecommand{\NWtxtFileDefBy}{File defined by}
\providecommand{\NWsep}{${\diamond}$}
\providecommand{\NWlink}[2]{\hyperlink{#1}{#2}}
\providecommand{\NWtarget}[2]{% move baseline up by \baselineskip 
  \raisebox{\baselineskip}[1.5ex][0ex]{%
    \mbox{%
      \hypertarget{#1}{%
        \raisebox{-1\baselineskip}[0ex][0ex]{%
          \mbox{#2}%
}}}}}
% -------------------------------------------------------------------------

\documentclass[11pt,oneside]{article}	%use"amsart"insteadof"article"forAMSLaTeXformat
\usepackage{geometry}		%Seegeometry.pdftolearnthelayoutoptions.Therearelots.
\geometry{letterpaper}		%...ora4paperora5paperor...
%\geometry{landscape}		%Activateforforrotatedpagegeometry
%\usepackage[parfill]{parskip}		%Activatetobeginparagraphswithanemptylineratherthananindent
\usepackage{graphicx}				%Usepdf,png,jpg,orepsßwithpdflatex;useepsinDVImode
								%TeXwillautomaticallyconverteps-->pdfinpdflatex		
\usepackage{amssymb}
\usepackage{hyperref}

%----macros begin-----------------------------------------------------------------------------------
\usepackage{graphicx}
\usepackage{color}
\usepackage{amsthm}

%\renewenvironment{Shaded}{\pause\begin{snugshade}}{\end{snugshade}}
\def\twocolumns#1#2{\begin{columns}
\begin{column}{0.5\linewidth}#1\end{column}
\begin{column}{0.5\linewidth}#2\end{column}
\end{columns}}
\def\mytwocolumns#1#2#3#4{\begin{columns}
\begin{column}{#1\linewidth}#2\end{column}
\begin{column}{#3\linewidth}#4\end{column}
\end{columns}}
\def\mythreecolumns#1#2#3#4#5#6{\begin{columns}
\begin{column}{#1\linewidth}#2\end{column}
\begin{column}{#3\linewidth}#4\end{column}
\begin{column}{#5\linewidth}#6\end{column}
\end{columns}}
\def\threecolumns#1#2#3{\begin{columns}
\begin{column}{0.33\linewidth}#1\end{column}
\begin{column}{0.33\linewidth}#2\end{column}
\begin{column}{0.33\linewidth}#3\end{column}
\end{columns}}
\def\fourcolumns#1#2#3#4{\begin{columns}%
\begin{column}{0.25\linewidth}#1\end{column}%
\begin{column}{0.25\linewidth}#2\end{column}%
\begin{column}{0.25\linewidth}#3\end{column}%
\begin{column}{0.25\linewidth}#4\end{column}%
\end{columns}}

\def\conv{\mbox{\textrm{conv}\,}}
\def\aff{\mbox{\textrm{aff}\,}}
\def\E{\mathbb{E}}
\def\R{\mathbb{R}}
\def\Z{\mathbb{Z}}
\def\tex{\TeX}
\def\latex{\LaTeX}
\def\v#1{{\bf #1}}
\def\p#1{{\bf #1}}
\def\T#1{{\bf #1}}

\def\vet#1{{\left(\begin{array}{cccccccccccccccccccc}#1\end{array}\right)}}
\def\mat#1{{\left(\begin{array}{cccccccccccccccccccc}#1\end{array}\right)}}

\def\lin{\mbox{\rm lin}\,}
\def\aff{\mbox{\rm aff}\,}
\def\pos{\mbox{\rm pos}\,}
\def\cone{\mbox{\rm cone}\,}
\def\conv{\mbox{\rm conv}\,}
\newcommand{\homog}[0]{\mbox{\rm homog}\,}
\newcommand{\relint}[0]{\mbox{\rm relint}\,}

%----macros end-----------------------------------------------------------------------------------


\title{The \texttt{smplxn} module
\footnote{This document is part of the framework~\cite{cclar-proj:2013:00}. \today}
}
\author{Alberto Paoluzzi}
%\date{}							%Activatetodisplayagivendateornodate

\begin{document}
\maketitle
\nonstopmode

\begin{abstract}
This module defines a minimal set of functions to generate a dimension-independent grid of simplices.
The name of the library was firstly used by our CAD Lab at University of Rome ``La Sapienza'' in years 1987/88 when we started working with dimension-independent simplicial complexes~\cite{Paoluzzi:1993:DMS:169728.169719}. This one in turn imports some functions from the \texttt{scipy} package and the geometric library \texttt{pyplasm}~\cite{}.
\end{abstract}

\tableofcontents\newpage

\section{Introduction}



\section{Signed (co)boundary matrices of a simplicial complex}
\label{simplicial}

\paragraph{Importing a library}
First of all, a modeling application having to deal with simplicial complexes must import the $Simple_X^n$ library, denoted \texttt{smplxn} in \texttt{python}. 

\begin{flushleft} \small
\begin{minipage}{\linewidth} \label{scrap1}
\protect\makebox[0ex][r]{\NWtarget{nuweb2a}{\rule{0ex}{0ex}}\hspace{1em}}$\langle\,$Inport the $Simple_X^n$ library\nobreak\ {\footnotesize 2a}$\,\rangle\equiv$
\vspace{-1ex}
\begin{list}{}{} \item
\mbox{}\verb@@\\
\mbox{}\verb@import sys@\\
\mbox{}\verb@sys.path.insert(0, 'lib/py/')@\\
\mbox{}\verb@from smplxn import *@\\
\mbox{}\verb@@{\NWsep}
\end{list}
\vspace{-1ex}
\footnotesize\addtolength{\baselineskip}{-1ex}
\begin{list}{}{\setlength{\itemsep}{-\parsep}\setlength{\itemindent}{-\leftmargin}}
\item \NWtxtMacroRefIn\ \NWlink{nuweb3c}{3c}\NWlink{nuweb4b}{, 4b}.
\end{list}
\end{minipage}\\[4ex]
\end{flushleft}
\begin{flushleft} \small
\begin{minipage}{\linewidth} \label{scrap2}
\protect\makebox[0ex][r]{\NWtarget{nuweb2b}{\rule{0ex}{0ex}}\hspace{1em}}$\langle\,$Inport a generic\nobreak\ {\footnotesize 2b}$\,\rangle\equiv$
\vspace{-1ex}
\begin{list}{}{} \item
$ module ${\NWsep}
\end{list}
\vspace{-1ex}
\footnotesize\addtolength{\baselineskip}{-1ex}
\begin{list}{}{\setlength{\itemsep}{-\parsep}\setlength{\itemindent}{-\leftmargin}}
\item {\NWtxtMacroNoRef}.
\end{list}
\end{minipage}\\[4ex]
\end{flushleft}
mod 

import sys
sys.path.insert(0, 'lib/py/')
import module as mod


\section{Test examples}

\subsection{Structured grid}

\subsubsection{2D example}

\paragraph{Generate a simplicial decomposition}
Then we generate and show a 2D decomposition of the unit square $[0,1]^2\subset\E^2$ into a $3\times 3$ grid of simplices (triangles, in this case), using the \texttt{simplexGrid} function, that returns a pair \texttt{(V,FV)}, made by the array \texttt{V} of vertices, and by the array \texttt{FV} of ``faces by vertex'' indices, that constitute a \emph{reduced} simplicial LAR of the $[0,1]^2$ domain. The computed \texttt{FV} array is then dispayed ``exploded'', being $ex,ey,ez$ the explosion parameters in the $x,y,z$ coordinate directions, respectively. Notice that the \texttt{MKPOLS} pyplasm primitive requires a pair \texttt{(V,FV)}, that we call a ``model'', as input --- i.e. a pair made by the array \texttt{V} of vertices, and by a zero-based array of array of indices of vertices. Elsewhere in this document we identified such a data structure as CSR$(M_d)$, for some dimension $d$. Suc notation stands for the Compressed Sparse Row representation of a binary characteristic matrix.

\begin{flushleft} \small
\begin{minipage}{\linewidth} \label{scrap3}
\protect\makebox[0ex][r]{\NWtarget{nuweb3a}{\rule{0ex}{0ex}}\hspace{1em}}$\langle\,$Generate a simplicial decomposition ot the $[0,1]^2$ domain\nobreak\ {\footnotesize 3a}$\,\rangle\equiv$
\vspace{-1ex}
\begin{list}{}{} \item
\mbox{}\verb@V,FV = simplexGrid([3,3])@\\
\mbox{}\verb@VIEW(EXPLODE(1.5,1.5,1.5)(MKPOLS((V,FV))))@\\
\mbox{}\verb@@{\NWsep}
\end{list}
\vspace{-1ex}
\footnotesize\addtolength{\baselineskip}{-1ex}
\begin{list}{}{\setlength{\itemsep}{-\parsep}\setlength{\itemindent}{-\leftmargin}}
\item \NWtxtMacroRefIn\ \NWlink{nuweb3c}{3c}.
\end{list}
\end{minipage}\\[4ex]
\end{flushleft}
\paragraph{Extract the $(d-1)$-faces}
Since the complex is simplicial, we can directly extract its facets (in this case the 1-faces, i.e. its edges) by invoking the \texttt{simplexFacets} function on the argument \texttt{FV}, so returning the array \texttt{EV} of ``edges by vertex'' indices. 

%-------------------------------------------------------------------------------
\begin{flushleft} \small
\begin{minipage}{\linewidth} \label{scrap4}
\protect\makebox[0ex][r]{\NWtarget{nuweb3b}{\rule{0ex}{0ex}}\hspace{1em}}$\langle\,$Extract the edges of the 2D decomposition\nobreak\ {\footnotesize 3b}$\,\rangle\equiv$
\vspace{-1ex}
\begin{list}{}{} \item
\mbox{}\verb@EV = simplexFacets(FV)@\\
\mbox{}\verb@ex,ey,ez = 1.5,1.5,1.5@\\
\mbox{}\verb@VIEW(EXPLODE(ex,ey,ez)(MKPOLS((V,EV))))@\\
\mbox{}\verb@@{\NWsep}
\end{list}
\vspace{-1ex}
\footnotesize\addtolength{\baselineskip}{-1ex}
\begin{list}{}{\setlength{\itemsep}{-\parsep}\setlength{\itemindent}{-\leftmargin}}
\item \NWtxtMacroRefIn\ \NWlink{nuweb3c}{3c}.
\end{list}
\end{minipage}\\[4ex]
\end{flushleft}
%-------------------------------------------------------------------------------

\paragraph{Export the executable file}
We are finally able to generate and output a complete test file, including the visualization expressions. This file can be executed by the \texttt{test} target of the \texttt{make} command.

%-------------------------------------------------------------------------------
\begin{flushleft} \small \label{scrap5}
\protect\makebox[0ex][r]{\NWtarget{nuweb3c}{\rule{0ex}{0ex}}\hspace{1em}}\verb@"test/py/test01.py"@\nobreak\ {\footnotesize 3c }$\equiv$
\vspace{-1ex}
\begin{list}{}{} \item
\mbox{}\verb@@\\
\mbox{}\verb@@\hbox{$\langle\,$Inport the $Simple_X^n$ library\nobreak\ {\footnotesize \NWlink{nuweb2a}{2a}}$\,\rangle$}\verb@@\\
\mbox{}\verb@@\hbox{$\langle\,$Generate a simplicial decomposition ot the $[0,1]^2$ domain\nobreak\ {\footnotesize \NWlink{nuweb3a}{3a}}$\,\rangle$}\verb@@\\
\mbox{}\verb@@\hbox{$\langle\,$Extract the edges of the 2D decomposition\nobreak\ {\footnotesize \NWlink{nuweb3b}{3b}}$\,\rangle$}\verb@@\\
\mbox{}\verb@@{\NWsep}
\end{list}
\vspace{-2ex}
\end{flushleft}
%-------------------------------------------------------------------------------

\subsubsection{3D example}

In this case we produce a $2\times 2\times 2$ grid of tetrahedra. The dimension (3D) of the model to be generated is inferred by the presence of 3 parameters in the parameter list of the \texttt{simplexGrid} function. 

%-------------------------------------------------------------------------------
\begin{flushleft} \small
\begin{minipage}{\linewidth} \label{scrap6}
\protect\makebox[0ex][r]{\NWtarget{nuweb3d}{\rule{0ex}{0ex}}\hspace{1em}}$\langle\,$Generate a simplicial decomposition ot the $[0,1]^3$ domain\nobreak\ {\footnotesize 3d}$\,\rangle\equiv$
\vspace{-1ex}
\begin{list}{}{} \item
\mbox{}\verb@V,CV = simplexGrid([2,2,2])@\\
\mbox{}\verb@VIEW(EXPLODE(1.5,1.5,1.5)(MKPOLS((V,CV))))@\\
\mbox{}\verb@@{\NWsep}
\end{list}
\vspace{-1ex}
\footnotesize\addtolength{\baselineskip}{-1ex}
\begin{list}{}{\setlength{\itemsep}{-\parsep}\setlength{\itemindent}{-\leftmargin}}
\item \NWtxtMacroRefIn\ \NWlink{nuweb4b}{4b}.
\end{list}
\end{minipage}\\[4ex]
\end{flushleft}
%-------------------------------------------------------------------------------

and repeat two times the facet extraction:

%-------------------------------------------------------------------------------
\begin{flushleft} \small
\begin{minipage}{\linewidth} \label{scrap7}
\protect\makebox[0ex][r]{\NWtarget{nuweb4a}{\rule{0ex}{0ex}}\hspace{1em}}$\langle\,$Extract the faces and edges of the 3D decomposition\nobreak\ {\footnotesize 4a}$\,\rangle\equiv$
\vspace{-1ex}
\begin{list}{}{} \item
\mbox{}\verb@@\\
\mbox{}\verb@FV = simplexFacets(CV)@\\
\mbox{}\verb@VIEW(EXPLODE(1.5,1.5,1.5)(MKPOLS((V,FV))))@\\
\mbox{}\verb@EV = simplexFacets(FV)@\\
\mbox{}\verb@VIEW(EXPLODE(1.5,1.5,1.5)(MKPOLS((V,EV))))@\\
\mbox{}\verb@@{\NWsep}
\end{list}
\vspace{-1ex}
\footnotesize\addtolength{\baselineskip}{-1ex}
\begin{list}{}{\setlength{\itemsep}{-\parsep}\setlength{\itemindent}{-\leftmargin}}
\item \NWtxtMacroRefIn\ \NWlink{nuweb4b}{4b}.
\end{list}
\end{minipage}\\[4ex]
\end{flushleft}
%-------------------------------------------------------------------------------

and finally export a new test file:

%-------------------------------------------------------------------------------
\begin{flushleft} \small \label{scrap8}
\protect\makebox[0ex][r]{\NWtarget{nuweb4b}{\rule{0ex}{0ex}}\hspace{1em}}\verb@"test/py/test02.py"@\nobreak\ {\footnotesize 4b }$\equiv$
\vspace{-1ex}
\begin{list}{}{} \item
\mbox{}\verb@@\\
\mbox{}\verb@@\hbox{$\langle\,$Inport the $Simple_X^n$ library\nobreak\ {\footnotesize \NWlink{nuweb2a}{2a}}$\,\rangle$}\verb@@\\
\mbox{}\verb@@\hbox{$\langle\,$Generate a simplicial decomposition ot the $[0,1]^3$ domain\nobreak\ {\footnotesize \NWlink{nuweb3d}{3d}}$\,\rangle$}\verb@@\\
\mbox{}\verb@@\hbox{$\langle\,$Extract the faces and edges of the 3D decomposition\nobreak\ {\footnotesize \NWlink{nuweb4a}{4a}}$\,\rangle$}\verb@@\\
\mbox{}\verb@@{\NWsep}
\end{list}
\vspace{-2ex}
\end{flushleft}
%-------------------------------------------------------------------------------


\subsection{Unstructured grid}


\subsubsection{2D example}


\subsubsection{3D example}



\bibliographystyle{amsalpha}
\bibliography{smplxn.bib}

\end{document}
