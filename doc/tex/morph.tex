% -------------------------------------------------------------------------
% ------ nuweb macros (redefine as desired, or omit with "nuweb -p") ------
% -------------------------------------------------------------------------
\providecommand{\NWtxtMacroDefBy}{Macro defined by}
\providecommand{\NWtxtMacroRefIn}{Macro referenced in}
\providecommand{\NWtxtMacroNoRef}{Macro never referenced}
\providecommand{\NWtxtDefBy}{Defined by}
\providecommand{\NWtxtRefIn}{Referenced in}
\providecommand{\NWtxtNoRef}{Not referenced}
\providecommand{\NWtxtFileDefBy}{File defined by}
\providecommand{\NWsep}{${\diamond}$}
\providecommand{\NWlink}[2]{\hyperlink{#1}{#2}}
\providecommand{\NWtarget}[2]{% move baseline up by \baselineskip 
  \raisebox{\baselineskip}[1.5ex][0ex]{%
    \mbox{%
      \hypertarget{#1}{%
        \raisebox{-1\baselineskip}[0ex][0ex]{%
          \mbox{#2}%
}}}}}
% -------------------------------------------------------------------------

\documentclass[11pt,oneside]{article}	%use"amsart"insteadof"article"forAMSLaTeXformat
\usepackage{geometry}		%Seegeometry.pdftolearnthelayoutoptions.Therearelots.
\geometry{letterpaper}		%...ora4paperora5paperor...
%\geometry{landscape}		%Activateforforrotatedpagegeometry
%\usepackage[parfill]{parskip}		%Activatetobeginparagraphswithanemptylineratherthananindent
\usepackage{graphicx}				%Usepdf,png,jpg,orepsßwithpdflatex;useepsinDVImode
								%TeXwillautomaticallyconverteps-->pdfinpdflatex		
\usepackage{amssymb}
\usepackage{hyperref}

%----macros begin---------------------------------------------------------------
\usepackage{color}
\usepackage{amsthm}

\def\conv{\mbox{\textrm{conv}\,}}
\def\aff{\mbox{\textrm{aff}\,}}
\def\E{\mathbb{E}}
\def\R{\mathbb{R}}
\def\Z{\mathbb{Z}}
\def\tex{\TeX}
\def\latex{\LaTeX}
\def\v#1{{\bf #1}}
\def\p#1{{\bf #1}}
\def\T#1{{\bf #1}}

\def\vet#1{{\left(\begin{array}{cccccccccccccccccccc}#1\end{array}\right)}}
\def\mat#1{{\left(\begin{array}{cccccccccccccccccccc}#1\end{array}\right)}}

\def\lin{\mbox{\rm lin}\,}
\def\aff{\mbox{\rm aff}\,}
\def\pos{\mbox{\rm pos}\,}
\def\cone{\mbox{\rm cone}\,}
\def\conv{\mbox{\rm conv}\,}
\newcommand{\homog}[0]{\mbox{\rm homog}\,}
\newcommand{\relint}[0]{\mbox{\rm relint}\,}

%----macros end-----------------------------------------------------------------

\title{Imaging Morphology with LAR
\footnote{This document is part of the \emph{Linear Algebraic Representation with CoChains} (LAR-CC) framework~\cite{cclar-proj:2013:00}. \today}
}
\author{Alberto Paoluzzi}
%\date{}							%Activatetodisplayagivendateornodate

\begin{document}
\maketitle
\nonstopmode

\begin{abstract}
In this module we aim to implement the four operators of mathematical morphology, i.e.~the \emph{dilation}, \emph{erosion}, \emph{opening} and \emph{closing} operators, by the way of matrix operations representing the linear operators---\emph{boundary} and \emph{coboundary}---over LAR. 
According to the multidimensional character of LAR, our implementation is dimension-independent.
In few words, it works as follows: (a)  the input is (the coordinate representation of) a $d$-chain $\gamma$; (b) compute its boundary $\partial_d(\gamma)$; (c) extract the maximal $(d-2)$-chain $\epsilon \subset \partial_d(\gamma)$; (d) consider the $(d-1)$-chain returned from its coboundary $\delta_{d-2}(\epsilon)$; (e) compute the $d$-chain $\eta := \delta_{d-1}(\delta_{d-2}(\epsilon)) \subset C_d$ \emph{without} performing the  $\mbox{mod\ 2}$ final transformation on the resulting coordinate vector, that would provide a zero result, according to the standard algebraic constraint $\delta\circ\delta=0$. It is easy to show that $\eta \equiv (\oplus \gamma) - (\ominus \gamma)$ provides the \emph{morphological gradient} operator. The four standard morphological operators are therefore  consequently computable.
\end{abstract}

\tableofcontents

\section{Test image generation}

Various methods for the input or the generation of a test image  are developed in the subsections of this section. The aim is to prepare a set of controlled test beds, used to check both the implementation and the working properties of our topological implementation of morphological operators. 


\subsection{Small 2D random binary image}

A small binary test image is generated here by using a random approach, both for the bulk structure and the small artefacts of the image.  

\paragraph{Generation of the gross image}
First we generate a 2D grid of squares by Cartesian product, and produce the bulk of the random image then used to test our approach to morphological operators via topological ones.


%import scipy
%scipy.misc.imsave('outfile.jpg', image_array)
%
%scipy.ndimage.imread(fname, flatten=False, mode=None)[source]
%
%rand(10, 10)
%

\begin{flushleft} \small
\begin{minipage}{\linewidth} \label{scrap1}
\protect\makebox[0ex][r]{\NWtarget{nuweb2a}{\rule{0ex}{0ex}}\hspace{1em}}$\langle\,$Generation of random image\nobreak\ {\footnotesize 2a}$\,\rangle\equiv$
\vspace{-1ex}
\begin{list}{}{} \item
\mbox{}\verb@import scipy.misc, numpy@\\
\mbox{}\verb@from numpy.random import randint@\\
\mbox{}\verb@rows, columns = 100,100@\\
\mbox{}\verb@rowSize, columnSize = 10,10@\\
\mbox{}\verb@@\\
\mbox{}\verb@random_array = randint(0, 255, size=(rowSize, columnSize))@\\
\mbox{}\verb@image_array = numpy.zeros((rows, columns))@\\
\mbox{}\verb@for i in range(rowSize):@\\
\mbox{}\verb@   for j in range(columnSize):@\\
\mbox{}\verb@      for h in range(i*rowSize,i*rowSize+rowSize): @\\
\mbox{}\verb@         for k in range(j*columnSize,j*columnSize+columnSize):@\\
\mbox{}\verb@            if random_array[i,j] < 127:@\\
\mbox{}\verb@               image_array[h,k] = 0 @\\
\mbox{}\verb@            else: @\\
\mbox{}\verb@               image_array[h,k] = 255@\\
\mbox{}\verb@scipy.misc.imsave('./outfile.png', image_array)@\\
\mbox{}\verb@@{\NWsep}
\end{list}
\vspace{-1ex}
\footnotesize\addtolength{\baselineskip}{-1ex}
\begin{list}{}{\setlength{\itemsep}{-\parsep}\setlength{\itemindent}{-\leftmargin}}
\item {\NWtxtMacroNoRef}.
\end{list}
\end{minipage}\\[4ex]
\end{flushleft}
\paragraph{Generation of random artefacts upon the image}

Then random noise is added to the previously generated image, in order to produce artifacts at the pixel scale. 

\begin{flushleft} \small
\begin{minipage}{\linewidth} \label{scrap2}
\protect\makebox[0ex][r]{\NWtarget{nuweb2b}{\rule{0ex}{0ex}}\hspace{1em}}$\langle\,$Generation of random artifacts\nobreak\ {\footnotesize 2b}$\,\rangle\equiv$
\vspace{-1ex}
\begin{list}{}{} \item
\mbox{}\verb@noiseFraction = 0.1@\\
\mbox{}\verb@noiseQuantity = rows*columns*noiseFraction@\\
\mbox{}\verb@k = 0@\\
\mbox{}\verb@while k < noiseQuantity:@\\
\mbox{}\verb@   i,j = randint(rows),randint(columns)@\\
\mbox{}\verb@   if image_array[i,j] == 0: image_array[i,j] = 255@\\
\mbox{}\verb@   else: image_array[i,j] = 0@\\
\mbox{}\verb@   k += 1@\\
\mbox{}\verb@scipy.misc.imsave('./outfile.png', image_array)@\\
\mbox{}\verb@@{\NWsep}
\end{list}
\vspace{-1ex}
\footnotesize\addtolength{\baselineskip}{-1ex}
\begin{list}{}{\setlength{\itemsep}{-\parsep}\setlength{\itemindent}{-\leftmargin}}
\item {\NWtxtMacroNoRef}.
\end{list}
\end{minipage}\\[4ex]
\end{flushleft}
\section{Selection of an image segment}

In this section we implement several image segmentation and selection of a segment methods. The first and simplest method is the selection of the portion of a binary image contained within a (mobile) image window.

\subsection{Selection of a test chain}

Here we select the (white) sub-image contained in a given image window, and compute the coordinate representation of the test sub-image.

\paragraph{}


\bibliographystyle{amsalpha}
\bibliography{morph}

\end{document}
