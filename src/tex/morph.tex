\documentclass[11pt,oneside]{article}	%use"amsart"insteadof"article"forAMSLaTeXformat
\usepackage{geometry}		%Seegeometry.pdftolearnthelayoutoptions.Therearelots.
\geometry{letterpaper}		%...ora4paperora5paperor...
%\geometry{landscape}		%Activateforforrotatedpagegeometry
%\usepackage[parfill]{parskip}		%Activatetobeginparagraphswithanemptylineratherthananindent
\usepackage{graphicx}				%Usepdf,png,jpg,orepsßwithpdflatex;useepsinDVImode
								%TeXwillautomaticallyconverteps-->pdfinpdflatex		
\usepackage{amssymb}
\usepackage{hyperref}

%----macros begin---------------------------------------------------------------
\usepackage{color}
\usepackage{amsthm}

\def\conv{\mbox{\textrm{conv}\,}}
\def\aff{\mbox{\textrm{aff}\,}}
\def\E{\mathbb{E}}
\def\R{\mathbb{R}}
\def\Z{\mathbb{Z}}
\def\tex{\TeX}
\def\latex{\LaTeX}
\def\v#1{{\bf #1}}
\def\p#1{{\bf #1}}
\def\T#1{{\bf #1}}

\def\vet#1{{\left(\begin{array}{cccccccccccccccccccc}#1\end{array}\right)}}
\def\mat#1{{\left(\begin{array}{cccccccccccccccccccc}#1\end{array}\right)}}

\def\lin{\mbox{\rm lin}\,}
\def\aff{\mbox{\rm aff}\,}
\def\pos{\mbox{\rm pos}\,}
\def\cone{\mbox{\rm cone}\,}
\def\conv{\mbox{\rm conv}\,}
\newcommand{\homog}[0]{\mbox{\rm homog}\,}
\newcommand{\relint}[0]{\mbox{\rm relint}\,}

%----macros end-----------------------------------------------------------------

\title{Imaging Morphology with LAR
\footnote{This document is part of the \emph{Linear Algebraic Representation with CoChains} (LAR-CC) framework~\cite{cclar-proj:2013:00}. \today}
}
\author{Alberto Paoluzzi}
%\date{}							%Activatetodisplayagivendateornodate

\begin{document}
\maketitle
\nonstopmode

\begin{abstract}
In this module we aim to implement the four operators of mathematical morphology, i.e.~the \emph{dilation}, \emph{erosion}, \emph{opening} and \emph{closing} operators, by the way of matrix operations representing the linear operators---\emph{boundary} and \emph{coboundary}---over LAR. 
According to the multidimensional character of LAR, our implementation is dimension-independent.
In few words, it works as follows: (a)  the input is (the coordinate representation of) a $d$-chain $\gamma$; (b) compute its boundary $\partial_d(\gamma)$; (c) extract the maximal $(d-2)$-chain $\epsilon \subset \partial_d(\gamma)$; (d) consider the $(d-1)$-chain returned from its coboundary $\delta_{d-2}(\epsilon)$; (e) compute the $d$-chain $\eta := \delta_{d-1}(\delta_{d-2}(\epsilon)) \subset C_d$ \emph{without} performing the  $\mbox{mod\ 2}$ final transformation on the resulting coordinate vector, that would provide a zero result, according to the standard algebraic constraint $\delta\circ\delta=0$. It is easy to show that $\eta \equiv (\oplus \gamma) - (\ominus \gamma)$ provides the \emph{morphological gradient} operator. The four standard morphological operators are therefore  consequently computable.
\end{abstract}

\tableofcontents

\section{Test image generation}

Various methods for the input or the generation of a test image  are developed in the subsections of this section. The aim is to prepare a set of controlled test beds, used to check both the implementation and the working properties of our topological implementation of morphological operators. 


\subsection{Small 2D random binary image}

A small binary test image is generated here by using a random approach, both for the bulk structure and the small artefacts of the image.  

\paragraph{Generation of the gross image}
First we generate a 2D grid of squares by Cartesian product, and produce the bulk of the random image then used to test our approach to morphological operators via topological ones.


%import scipy
%scipy.misc.imsave('outfile.jpg', image_array)
%
%scipy.ndimage.imread(fname, flatten=False, mode=None)[source]
%
%rand(10, 10)
%

%-------------------------------------------------------------------------------
@d Generation of random image
@{import scipy.misc, numpy
from numpy.random import randint
rows, columns = 100,100
rowSize, columnSize = 10,10

random_array = randint(0, 255, size=(rowSize, columnSize))
image_array = numpy.zeros((rows, columns))
for i in range(rowSize):
	for j in range(columnSize):
		for h in range(i*rowSize,i*rowSize+rowSize): 
			for k in range(j*columnSize,j*columnSize+columnSize):
				if random_array[i,j] < 127:
					image_array[h,k] = 0 
				else: 
					image_array[h,k] = 255
scipy.misc.imsave('./outfile.png', image_array)
@}
%-------------------------------------------------------------------------------

\paragraph{Generation of random artefacts upon the image}

Then random noise is added to the previously generated image, in order to produce artifacts at the pixel scale. 

%-------------------------------------------------------------------------------
@d Generation of random artifacts
@{noiseFraction = 0.1
noiseQuantity = rows*columns*noiseFraction
k = 0
while k < noiseQuantity:
	i,j = randint(rows),randint(columns)
	if image_array[i,j] == 0: image_array[i,j] = 255
	else: image_array[i,j] = 0
	k += 1
scipy.misc.imsave('./outfile.png', image_array)
@}
%-------------------------------------------------------------------------------


\section{Selection of an image segment}

In this section we implement several methods for image segmentation and segment selection. The first and simplest method is the selection of the portion of a binary image contained within a (mobile) image window.

\subsection{Selection of a test chain}

Here we select the (white) sub-image contained in a given image window, and compute the coordinate representation of the test sub-image.

\paragraph{Image window}

A window within a $d$-image is defined by $2\times d$ integer numbers (2 multi-indices), corresponding to the window  \texttt{minPoint} (minimum indices) and to the window \texttt{maxPoint} (maximum indices). A list of multi-index tuples, contained in the \texttt{window} variable, is generated by the macro \emph{Generation of multi-index window} below.

%-------------------------------------------------------------------------------
@d Generation of multi-index window
@{from pyplasm import *
minPoint, maxPoint = (20,20), (40,30)
indexRanges = zip(minPoint,maxPoint)
window = CART([range(min,max) for min,max in indexRanges])
@}
%-------------------------------------------------------------------------------

\paragraph{From window multi-indices to chain coordinates}

The set of tuples within the \texttt{window} is here mapped to the corresponding set of (single) integers associated to the low-level image elements (pixels or voxels, depending on the image dimension and shape), denoted \texttt{windowChain}. Such total chain of the image \texttt{window} is then filtered to contain the only coordinates of \emph{white} image elements within the window, and returned as the set of integer cell indices \texttt{segmentChain}.

%-------------------------------------------------------------------------------
@d Window-to-chain mapping
@{imageShape = [rows,columns]
d = len(imageShape)
weights = [PROD(imageShape[(k+1):]) for k in range(d-1)]+[1]
imageCochain = image_array.reshape(PROD(imageShape))
windowChain = [INNERPROD([index,weights]) for index in window]
segmentChain = [cell for cell in windowChain if imageCochain[cell]==255]
@}
%-------------------------------------------------------------------------------

\subsection{Show segment chain from binary image}

Now we need to show visually the selected \texttt{segmentChain}, by change the color of its cells from white (255) to middle grey (127). Just remember that \texttt{imageCochain} is the linear representation of the image, with number of cells equal to \texttt{PROD(imageShape)}. Then the modified image is restored within \texttt{image\_array}, and is finally exported to a \texttt{.png} image file.

%-------------------------------------------------------------------------------
@d Change chain color to grey
@{for cell in segmentChain: imageCochain[cell] = 127
image_array = imageCochain.reshape(imageShape)
scipy.misc.imsave('./outfile.png', image_array)
@}
%-------------------------------------------------------------------------------


\paragraph{Test example}

The macros previously defined are here composed to generate a random black and white image, with a \emph{image segment} (in a fixed position window within the image) extracted, colored in middle grey, and exported to an image file.  

%------------------------------------------------------------------
@o test/py/morph/test01.py
@{@< Import the module @(largrid@) @>
@< Generation of random image @>
@< Generation of random artifacts @>
@< Generation of multi-index window @>
@< Window-to-chain mapping @>
@< Change chain color to grey @>
@< Pyplasm visualisation of an image chain @>
@}
%------------------------------------------------------------------

\section{Construction of (co)boundary operators}

A $d$-image is a \emph{cellular $d$-complex} where cells are $k$-cuboids ($0\leq k\leq d$), i.e.~Cartesian products of a number $k$ of 1D intervals, embedded in $d$-dimensional Euclidean space. 

A direct construction of cuboidal complexes is offered in \texttt{larcc} by the \texttt{largrid} module. 
The \texttt{visImageChain} function given by the macro \emph{Visualisation of an image chain} below. 


\subsection{Visualisation of an image chain}

%-------------------------------------------------------------------------------
@d Pyplasm visualisation of an image chain
@{def visImageChain (imageShape,chain):
	model = larCuboids(imageShape)
	imageVerts = model[0]
	imageLAR = model[1]
	chainLAR = [cell for k,cell in enumerate(imageLAR) if k in chain]
	return imageVerts,chainLAR
	
if __name__== "__main__":
	model = visImageChain (imageShape,segmentChain)
	VIEW(EXPLODE(1.2,1.2,1.2)(MKPOLS(model)))
	d = len(imageShape) - 1
	larGridSkeleton(imageShape)(d)
@}
%-------------------------------------------------------------------------------




%===============================================================================
\appendix
\section{Utilities}

\subsection{Importing a generic module}
First we define a parametric macro to allow the importing of \texttt{larcc} modules from the project repository \texttt{lib/py/}. When the user needs to import some project's module, she may call this macro as done in Section~\ref{sec:lar2psm}.
%------------------------------------------------------------------
@d Import the module
@{import sys
sys.path.insert(0, 'lib/py/')
import @1
from @1 import *
@}
%------------------------------------------------------------------

\paragraph{Importing a module} A function used to import a generic \texttt{lacccc} module within the current environment is also useful.
%------------------------------------------------------------------
@d Function to import a generic module
@{def importModule(moduleName):
	@< Import the module @(moduleName@) @>
@| importModule @}
%------------------------------------------------------------------


\bibliographystyle{amsalpha}
\bibliography{morph}

\end{document}
